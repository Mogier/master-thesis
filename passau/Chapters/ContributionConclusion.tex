%!TEX root = ../thesis.tex

\chapter{Conclusion} % Main chapter title

\label{chapter:ContributionConclusion} % Change X to a consecutive number; for referencing this chapter elsewhere, use \ref{ChapterX}

\lhead{Chapter 12 \emph{Contribution Conclusion}} % Change X to a consecutive number; this is for the header on each page - perhaps a shortened title

This study dealt with the subject of semantic enrichment of images' annotation with the help of the semantic web. To do so we reviewed the literature about the semantic web and its technologies as well as the existing approaches for semantic enrichment. We extracted good ideas from these approaches and then proposed a prototype which detect candidates tags for an image given an initial set of annotations.\\

Our study showed that several semantic resources can be used to achieve our task but no existing paper already tried to concatenate their ontologies. A novel methodology for candidates detection was proposed, making use of two ontologies gathered in one graph. We showed that using multiple resources could be quite beneficial because of the fact that, even if their ontologies are different (and so propose different candidates), they share concepts which allow us to build equivalence bonds. Additionally, we used a database of Flickr's images to test our assumptions and we proposed two evaluation's methods. We deduced that the generalization of initial tags are mathematically quite good candidates but aren't the best ones since they don't directly speak to the user.\\

As presented in the previous chapter, the work done could be further investigated to enhance the candidates' relevancy and to adapt it to other media documents. 