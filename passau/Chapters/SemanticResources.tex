%!TEX root = ../thesis.tex

\chapter{Semantic Web} % Main chapter title

\label{chapter:SemanticWebResources} % Change X to a consecutive number; for referencing this chapter elsewhere, use \ref{ChapterX}

\lhead{Chapter 2 \emph{Semantic Web}} % Change X to a consecutive number; this is for the header on each page - perhaps a shortened title

Our study is focus on semantic enrichment of an initial set of keywords which can be organized as sentences or not. It is important to first understand what is a semantic concept and how concepts are organized into ontologies.\\ 

In this section, we will present some general notions about semantic concepts and review several semantic resources, their hierarchical structures and how we access them.

\section{Generalities} % (fold)
\label{sec:generalities}
Linguistic semantics is the study of meaning that is used for understanding human expression through language. It is easy for two human-being to communicate (given that they speak the same language) and to understand what their partner say even if he's using a tricky turn of phrase. However, this task becomes way more difficult when it comes to the comprehension of the human language by a machine. How can the computer guess that \say{I am totally dead} means in fact \say{I am really tired} and that the speaker isn't actually dead ? Machines need structured resources to understand us and the Semantic Web is one of them.\\

The notion of \say{Semantic Web} has been mentioned for the first time by Berners-Lee et al in \cite{berners2001semantic}. In this paper, they describe it as a Web which is readable by machines in opposite of most of Web's content which were designed for humans to read. The Semantic Web isn't a separate Web but an extension of the current one which will bring structure to the meaningful content of Web pages.\\

Two main technologies are used for the development of the Semantic Web : eXtensible Markup Language ( short XML) and the Resource Description Framework (short RDF). XML allows everyone to create their own tags and to arbitrary structure their documents but gives no information about what this structure means. Meaning is provided by RDF which stores it in sets of triples which are composed of a subject, a predicate and an object. Those three components can be related to the subject, the verb and object of an elementary sentence. In \cite{miller1998introduction}, Miller present a short introduction to the RDF standard and precise that a \say{Resource} can be any object which is uniquely identifiable by a Uniform Resource Identifier (URI).\\

The third basic component of the Semantic Web are collections of information called ontologies. An ontology is, in computer science, a document which defines the relations among concepts. Basically, Web ontologies are composed of a taxonomy, which defines classes of objects and their relations, and a set of inference rules.\\

In addition, the The New Oxford Dictionary of English defines the notion of \say{semantic concept} as : \emph{An idea or thought that corresponds to some distinct entity or class of entities, or to its essential features, or determines the application of a term, and thus plays a part in the use of reason or language}.\\

Given those basic notions, we will now further detail four semantic web resources, their taxonomies and review some of their usage found in the literature.
% section generalities (end)

\section{Semantic Resources} % (fold)
\label{sec:semantic_resources}
\subsection{DBpedia} % (fold)
\label{sub:dbpedia}

DBpedia\footnote{http://wiki.dbpedia.org/} is a project originally launched by two German universities (Berlin and Leipzig) and backed by an important community. It explores Wikipedia\footnote{https://en.wikipedia.org/} and extract information from it which results on the creation of a multilingual, large-scale knowledge base. The extraction framework, all the available end-points as well as some facts and figures about the project are presented in \cite{lehmann2014dbpedia}.\\

DBpedia's ontology is based on classes (320 items) which form a subsumption hierarchy, the root element being owl:Thing, with a maximal depth of 5\footnote{Complete classes tree : http://mappings.dbpedia.org/server/ontology/classes/}. Theses classes are described by a total of 1650 different properties, forming a large set of RDF triples (580 million extracted from the English version of Wikipedia).\\

Even though DBpedia is now a worldwide project and provides pages in 125 languages, the English one is still the most represented. We can indeed find 4.58 million of things\footnote{http://wiki.dbpedia.org/about/facts-figures} including 1,445,000 instances of the class \textit{Person}, 735,000 places \textit{Place}, 251,000 \textit{Species} \dots The number of instances described in this language is about three time larger than the second and third language (French and German). \\

As well as any RDF-structured dataset, DBpedia can be requesting with SPARQL (which is an recursive acronym : SPARQL Protocol and RDF Query Language) queries. SPARQL allows the user to search, add, modify or delete RDF data available on the Internet, see \cite{prud2008sparql} for more details about the language. \\

\lstinputlisting[language=SPARQL,
	caption=SPARQL Query : Search classes,
	label={code:findClass}]
	{./Primitives/findClass.sparql}

\lstinputlisting[language=SPARQL,
	caption=SPARQL Query : Search superclasses,
	label={code:findSuperClass}]
	{./Primitives/findSuperClass.sparql}

Codes \ref{code:findClass} and \ref{code:findSuperClass} present two simple and generic SPARQL search queries wich return respectively the class(es) of a given entity E and the superclass(es) of a given class C using the \textit{type} and \textit{subClassOf} predicates.\\

DBpedia also provides useful web services and HTTP endpoints. DBpedia Spotlight, which highlight DBpedia concepts in an input text is described in \cite{mendes2011dbpedia} and further details about disambiguation using this service are presented in subsection \ref{sec:dbpedia_spotlight}. The official DBpedia SPARQL endpoint\footnote{http://dbpedia.org/sparql} allows the user to send SPARQL queries to the online Virtuoso Triple Store by using the browser interface or by sending a HTPP request. We learn in \cite{lehmann2014dbpedia} that the average amount of hits per day of this endpoint is of 2,910,410 for the 3.8 dataset version. \\

We find lot of papers in the literature which mention DBpedia as an asset for systems based on the Semantic Web. If some of those papers are still in the research field, like \cite{passant2010dbrec} which proposes a music recommendation system built on top of DBpedia, others present \say{real} applications which are currently in use. We can cite for instance \cite{kobilarov2009media} which describes how the BBC\footnote{http://www.bbc.co.uk/} uses DBpedia to backbone its publications. 
% subsection dbpedia (end)

\subsection{Geonames} % (fold)
\label{sub:geonames}

GeoNames\footnote{http://www.geonames.org/} is a geographical database which contains over 10 million geographical names. The most documented countries\footnote{http://www.geonames.org/statistics/} are the United States of America (2,203,094 names), Norway (600,008) and China (526,456). All resources are categorized into one out of nine classes and further subcategorized into one out of 645 codes\footnote{http://www.geonames.org/export/codes.html}. Obviously, the root element of Geonames' hierarchy is mother earth.\\

Like DBpedia, the Geonames' ontology makes possible the addition of new resources to the World Wide Web. However, Geonames distinguish the features' Concept from the RDF document about it. In consequence, each feature possesses two representation in Geonames. See the two following URIs as example :
\begin{description}
\item[URI1] http://sws.geonames.org/8015555/
\item[URI2] http://sws.geonames.org/8015555/about.rdf
\end{description}
The first one stands for the \say{Notre Dame de Fourvière} church in Lyon, France. This URI is used when one wants to refers to the church itself. The second URI is the RDF document with what Geonames knows about the church, its latitude and longitude for instance, or some nearby locations.\\

In order to allow the user to browse its \emph{Tremendous set of data} (Sir T. Berners-Lee), Geonames proposes a lot of REST-based web-services. In the case of image annotation, one in particular could be useful. Given that lot of recent numeric images contain EXIF data which embed GPS information such as the longitude and the latitude (see \cite{tevsic2005metadata} for further details about the EXIF format), one could add geographic tags to the picture with the help of the \textit{findNearbyPlaceName} service. Here is an example of usage :\\
\underline{Query} : findNearbyPlaceName?lat=48.566\&lng=13.43\&username=demo 
\lstinputlisting[language=XML,
	caption=XML response,
	label={code:passauXML}]
	{./Primitives/passau.xml}

Given its specific domain, we found less use-cases of Geonames in the literature than DBpedia's. Some interesting papers have however been presented, like \cite{stern2010detection} which use several semantic resources including Geonames in order to detect named entities in \say{Agence France Presse} (AFP) wires.
% subsection geonames (end)

\subsection{WordNet} % (fold)
\label{sub:wordnet}

WordNet\footnote{https://wordnet.princeton.edu/} is a lexical database of English which has been presented for the first time in 1995 in \cite{miller1995wordnet}. It is hosted by the Princeton University, currently running version 3.1 but there are no current plan for a future release due to limited staffing.\\

Its structure is based on the concept of \say{synset} (synonym set), a set cognitive synonyms. WordNet distinguish among Types (common nouns, verbs\dots) and Instances (specific persons\dots). Synsets are interliked using conceptual, semantic and lexical relations. The hierarchy is built by the use of the super-subordinate relation (or hyperonymy, hyponymy in WordNet's jargon). These relations implements the two directions of the \say{IS-A} expression. For instance, \textit{fruit} is \textbf{hyperonym} of \textit{apple} and \textit{horse} is a \textbf{hyponym} of \textit{animal}, the root element being \say{entity}. Other relations are also provided, like the antonymy (opposite of synonymy) or the meronymy and its opposite holonymy which implements the \say{IS-PART-OF} relation : \textit{finger} is a \textbf{meronym} of \textit{hand}. All these relations are transitive.\\

This resource is useful if we are searching for entities. Since the maximal depth is of the ontology is of 16, the leafs are very detailed nous (tsetse-fly, Yukon white birch, \dots) but it also contains more general concepts (vehicle, animal, \dots). WordNet contains at the moment 155,287 unique strings incuding 117,798 nouns.\\ 

It exists several ways to browse this resource. An online interface allows the user to manually query the dataset and to navigate in it through hyperlinks. For software and research purposes, the user has to download one of the released version of WordNet's dataset as well as a specific library according to the code language he's using.\\

Due to the lot of different relations between its synsets, WordNet has mostly been used for measuring semantic distances (see \cite{budanitsky2001semantic}, \cite{richardson1994using} and \cite{agirre2009study}) or to disambiguate texts (in \cite{resnik1995disambiguating}, \cite{voorhees1993using} and \cite{banerjee2002adapted}).\\
% subsection wordnet (end)

\subsection{ImageNet} % (fold)
\label{sub:imagenet}

ImageNet\footnote{http://www.image-net.org/} is an image database built on the WordNet hierarchy. It has been launch in 2009 and presented in \cite{deng2009imagenet}. Each of WordNet's synsets is depicted in ImageNet by a set of pictures (more than 500 in average). At the moment 14,197,122 images are referenced.\\

The stated aim of ImageNet is to serve as an asset for pedagogical and research purposes. It has for instance been used in \cite{deselaers2011visual} to measure the correlation between visual and semantic similarities or in a kindergarten in Canada to provide matching exercises to the children.\\

Despite the fact this resource is not as used as the previous ones, it was important to cite it in order to highlight that semantic concepts and images can be related even tough it's they are different kinds of medium.  
% section imagenet (end)

% section semantic_resources (end)
