%!TEX root = ../thesis.tex

\chapter{Disambiguation} % Main chapter title

\label{chapter:Disambiguation} % Change X to a consecutive number; for referencing this chapter elsewhere, use \ref{ChapterX}

\lhead{Chapter 3 \emph{Disambiguation}} % Change X to a consecutive number; this is for the header on each page - perhaps a shortened title

Let's remember what we say about semantics in \ref{sec:generalities} : it is the study of meaning. The question that we will study here is the following : What happens when a word has several meanings ?\\
The disambiguation process aims to determine which one of the meanings is relevant in a specific context. In our case, we want to disambiguate the initial tags from a picture in order to propose relevant candidates. To achieve this, several approaches are presented with usages from the literature. Finally, we discuss the its interest in the image annotation process and the issues it may rise. 

\section{How does it work} % (fold)
\label{sec:how_does_it_work}

% section how_does_it_work (end)

\section{Practical example : DBpedia Spotlight} % (fold)
\label{sec:dbpedia_spotlight}

% section dbpedia_spotlight (end)